\ifdefined\handout
  \documentclass[handout]{beamer}
\else
  \documentclass{beamer}
\fi

\usetheme{boxes}
\definecolor{beamer@structure@color}{rgb}{0,0,0}

\usecolortheme{structure}

\setbeamertemplate{footline}[frame number]
\setbeamertemplate{frametitle}{\color{black}
\def\myhrulefill{\leavevmode\leaders\hrule height 1pt\hfill\kern 0pt}
\headingfont\insertframetitle\par\vskip-8pt\myhrulefill}

\usepackage{tikz-cd}
\usetikzlibrary{arrows}
\usetikzlibrary{calc}
\usetikzlibrary{babel}

\newcommand*{\longhookrightarrow}{\ensuremath{\lhook\joinrel\relbar\joinrel\rightarrow}}
\newcommand{\NN}{\mathbb{N}}
\newcommand{\ZZ}{\mathbb{Z}}
\newcommand{\QQ}{\mathbb{Q}}
\newcommand{\RR}{\mathbb{R}}
\newcommand{\CC}{\mathbb{C}}
\newcommand{\FF}{\mathbb{F}}
\newcommand{\isom}{\cong}
\DeclareMathOperator{\Gal}{Gal}
\renewcommand{\Re}{\operatorname{Re}}
\newcommand{\dfn}{\mathrel{\mathop:}=}
\DeclareMathOperator{\Hom}{Hom}
\DeclareMathOperator{\rk}{rk}
\DeclareMathOperator{\Spec}{Spec}

\setbeamertemplate{navigation symbols}{}

\usepackage{array}
\newcolumntype{x}[1]{>{\centering\hspace{0pt}}p{#1}}
\definecolor{shadecolor}{rgb}{0.89,0.89,0.89}
\usepackage{colortbl}

\newcommand{\term}{\textbf}

\usepackage{mathspec}
\setsansfont[BoldFont={PT Sans Bold}, ItalicFont={PT Sans Italic}]{PT Sans}
\setmathrm[BoldFont={PT Sans Bold}, ItalicFont={PT Sans Italic}]{PT Sans}
\newfontfamily\headingfont[]{PT Sans Bold}

\begin{document}

% % % % % % % % % % % % % % % % % % % % % % % % % % % % % % % % % % % % % % % % % % % % %

\begin{frame}[noframenumbering]
  \headingfont
  \begin{center}
    {\huge Weil-étale cohomology for n < 0

    }

    \vspace{3em}

    {\large Alexey Beshenov}

    \vspace{0.20em}

    (CIMAT, Guanajuato)

    \vspace{3em}

    November 29, 2019

    \vspace{1em}

    First IMSA Conference\\
    Centro de Colaboración Samuel Gitler / CINVESTAV

  \end{center}
\end{frame}

% % % % % % % % % % % % % % % % % % % % % % % % % % % % % % % % % % % % % % % % % % % % %

\begin{frame}
  \frametitle{Arithmetic zeta-functions (Serre, 1965)}

  \onslide<2->\[ \begin{tikzcd}
      X \ar{d}{\small\begin{tabular}{l} separated, \\ finite type \end{tabular}} \\
      \Spec \ZZ
    \end{tikzcd} \]

  \vspace{\fill}

  \onslide<3->\[ \zeta_X (s) \dfn \prod_{\substack{x \in X \\ \text{closed}}} \frac{1}{1 - \# (\mathcal{O}_{X,x}/\mathfrak{m})^{-s}}. \quad (\Re s > \dim X) \]

  \vspace{\fill}

  \onslide<4->{\textbf{Conjecture}: meromorphic continuation to $s \in \CC$}.
\end{frame}

% % % % % % % % % % % % % % % % % % % % % % % % % % % % % % % % % % % % % % % % % % % % %

\begin{frame}
  \frametitle{Extensively studied cases}

  \begin{itemize}
  \item<2-> \textbf{Riemann}:
    $\zeta (s) = \prod_p \frac{1}{1 - p^{-s}} = \zeta_{\Spec \ZZ} (s)$.

  \item<3-> \textbf{Dedekind}: $\zeta_F (s) = \zeta_{\Spec \mathcal{O}_F} (s)$
    for a number field $F/\QQ$.

  \item<4-> \textbf{Hasse--Weil}: $X/\FF_q$, then
    $$\zeta_X (s) = Z_X (q^{-s}),$$
    where
    $$Z_X (t) = \exp \left(\sum_{m \ge 1} \frac{\# X (\FF_{q^m})}{m}\,t^m\right) \stackrel{\text{Dwork}}{\in} \QQ (t).$$
    (Cf. Weil conjectures.)
  \end{itemize}
\end{frame}

% % % % % % % % % % % % % % % % % % % % % % % % % % % % % % % % % % % % % % % % % % % % %

\begin{frame}
  \frametitle{Special values}

  \begin{itemize}
  \item<2-> Fix $n \in \ZZ$.

  \item<3-> $d_n \dfn$ \term{vanishing order} of $\zeta_X (s)$ at $s = n$.

  \item<4-> \term{Special value} (leading Taylor coefficient) at $s = n$:
    $$\zeta_X^* (n) \dfn \lim_{s \to n} (s-n)^{-d_n}\,\zeta_X (s).$$
  \end{itemize}
\end{frame}

% % % % % % % % % % % % % % % % % % % % % % % % % % % % % % % % % % % % % % % % % % % % %

\begin{frame}
  \frametitle{Classical motivation: class number formula}

  \begin{itemize}
  \item<2-> Let $X = \Spec \mathcal{O}_F$ and $n = 0$.

  \item<3-> Zero of order $d_0 = r_1 + r_2 - 1$,

    where $r_1 \dfn \#$ real places, $2 r_2 \dfn \#$ complex places.

  \item<4->Special value
    $\zeta_F^* (0) = - \cfrac{\# H^1 (\Spec \mathcal{O}_F, \mathbb{G}_m)}{\# H^0 (\Spec \mathcal{O}_F, \mathbb{G}_m)_{tors}}\,R_F$,

    $R_F \dfn$ \textbf{Dirichlet regulator} $\in \RR$.

  \item<5-> Formulas for other $n \in \ZZ$?
  \end{itemize}
\end{frame}

% % % % % % % % % % % % % % % % % % % % % % % % % % % % % % % % % % % % % % % % % % % % %

\begin{frame}
  \frametitle{Weil-étale cohomology (Lichtenbaum, 2000s)}

  \textbf{Conjectural cohomology theory}.

  \begin{itemize}
  \item<2-> Groups $H_{W,c}^i (X, \ZZ (n)) = H^i (R\Gamma_{W,c} (X, \ZZ (n)))$.

  \item<3-> Perfectness: finitely generated and $= 0$ for $|i| \gg 0$.

  \item<4-> Long exact sequence
    $$\cdots \to H_{W,c}^i (X, \ZZ (n)) \otimes \RR \to H_{W,c}^{i+1} (X, \ZZ (n)) \otimes \RR \to \cdots$$

  \item<5-> Knudsen--Mumford determinants $\Longrightarrow$ canonical
    isomorphism
    $$\lambda\colon \RR \xrightarrow{\isom} (\underbrace{\det\nolimits_\ZZ R\Gamma_{W,c} (X, \ZZ (n))}_{\text{free }\ZZ\text{-mod of rk }1}) \otimes \RR.$$

  \item<6-> $d_n \stackrel{???}{=} \sum_i (-1)^i \cdot i \cdot \rk_\ZZ H^i_{W,c} (X, \ZZ (n))$.

  \item<7-> $\lambda (\zeta_X^* (n)^{-1})\cdot \ZZ \stackrel{???}{=} \det\nolimits_\ZZ R\Gamma_{W,c} (X, \ZZ (n))$.
  \end{itemize}
\end{frame}

% % % % % % % % % % % % % % % % % % % % % % % % % % % % % % % % % % % % % % % % % % % % %

\begin{frame}
  \frametitle{Some work on Weil-étale cohomology}

  \begin{tabular}{lll}
    \onslide<2->{\textbf{Lichtenbaum}, 2005: & $X/\FF_q$ smooth} \\
    \onslide<2->{+ work by \textbf{Geisser}}\\
    \\
    \onslide<3->{\textbf{Lichtenbaum}, 2009: & $X = \Spec \mathcal{O}_F$} \\
    \\
    \onslide<4->{\textbf{Morin}, 2014: & $X/\ZZ$ proper, regular, & $n = 0$} \\
    \\
    \onslide<5->{\textbf{Flach, Morin}, 2018: & $X/\ZZ$ proper, regular, & $n \in \ZZ$} \\
    \\
    \onslide<6->{\textbf{---}, 2018: & $X/\ZZ$ any...} & \onslide<7->{$n < 0$} \\
  \end{tabular}
\end{frame}

% % % % % % % % % % % % % % % % % % % % % % % % % % % % % % % % % % % % % % % % % % % % %

\begin{frame}[plain]
  \headingfont

  \begin{center}
    {\huge From now on fix n < 0}
  \end{center}
\end{frame}

% % % % % % % % % % % % % % % % % % % % % % % % % % % % % % % % % % % % % % % % % % % % %

\begin{frame}
  \frametitle{Motivic cohomology $H^\bullet (X_\text{\it ét}, \ZZ^c (n))$}

  \begin{itemize}
  \item<2-> \textbf{Geisser}, 2010: \textbf{dualizing cycle complexes}
    $\ZZ^c (n)$.

    Complexes of abelian sheaves on $X_\text{\it ét}$.

  \item<3-> A variation of \textbf{Bloch's cycle complexes} (1986).

  \item<4-> Motivation: arithmetic duality theorems.

  \item<5-> Behaves as \textbf{Borel--Moore homology}: for
    $Z \to X \leftarrow U$
    $$R\Gamma (Z_\text{\it ét}, \ZZ^c (n)) \to R\Gamma (X_\text{\it ét}, \ZZ^c (n)) \to R\Gamma (U_\text{\it ét}, \ZZ^c (n)) \to [+1]$$

  \item<6-> Calculations: few and hard...

  \item<7-> \textbf{Conjecture} (Lichtenbaum):
    $H^i (X_\text{\it ét}, \ZZ^c (n))$ are finitely generated.
  \end{itemize}
\end{frame}

% % % % % % % % % % % % % % % % % % % % % % % % % % % % % % % % % % % % % % % % % % % % %

\begin{frame}
  \frametitle{Weil-étale complexes (after Flach and Morin)}

  \begin{itemize}
  \item<2-> Assuming Lichtenbaum's conjecture, there exists a perfect complex
    $R\Gamma_{W,c} (X, \ZZ (n))$.

  \item<3-> Splitting over $\RR$:
    \[ R\Gamma_{W,c} (X, \ZZ (n)) \otimes \RR \isom
      \left(\begin{array}{c}
              R\!\Hom (R\Gamma (X_\text{\it ét}, \ZZ^c (n)), \RR) [-1] \\
              \oplus \\
              R\Gamma_c (G_\RR, X (\CC), \RR (n)) [-1]
            \end{array}\right), \]
    $\RR (n) \dfn (2\pi i)^n\,\RR$, as a $G_\RR = \Gal (\CC/\RR)$-equivariant sheaf.

  \item<4-> Long exact sequence of $H^i_{W,c} (X, \ZZ (n)) \otimes \RR$: need a
    \term{regulator}.
  \end{itemize}
\end{frame}

% % % % % % % % % % % % % % % % % % % % % % % % % % % % % % % % % % % % % % % % % % % % %

\begin{frame}
  \frametitle{Regulator morphism}

  \begin{itemize}
  \item<2-> \textbf{Kerr--Lewis--Müller-Stach} (2006) $\Longrightarrow$ for
    $X_\CC$ is smooth and quasi-projective:
    $$Reg\colon R\Gamma (X_\text{\it ét}, \ZZ^c (n)) \to R\Gamma_{BM} (G_\RR, X (\CC), \RR (n)) [1].$$

  \item<3-> Note: as always, $n < 0$, this is why the RHS is simple.

  \item<4-> \textbf{Conjecture} (Beilinson): the dual
    $$Reg^\vee\colon R\Gamma_c (G_\RR, X (\CC), \RR (n)) [-1] \to R\!\Hom (R\Gamma (X_\text{\it ét}, \ZZ^c (n)), \RR)$$
    is a quasi-isomorphism.

  \item<5-> Splitting over $\RR$ + Beilinson's conjecture $\Longrightarrow$
    l.e.s.
    $$\cdots \to H_{W,c}^i (X, \ZZ (n)) \otimes \RR \to H_{W,c}^{i+1} (X, \ZZ (n)) \otimes \RR \to \cdots$$
  \end{itemize}
\end{frame}

% % % % % % % % % % % % % % % % % % % % % % % % % % % % % % % % % % % % % % % % % % % % %

\begin{frame}
  \frametitle{Main conjecture $\mathbf{C} (X,n)$}

  \begin{itemize}
  \item<2-> Assume...\\
    \hspace{1em}meromorphic continuation of $\zeta_X (s)$ around $s = n < 0$,\\
    \hspace{1em}$X_\CC$ is smooth quasi-projective,\\
    \hspace{1em}Lichtenbaum's and Beilinson's conjectures.

  \item<3-> \textbf{Then}
    \begin{align*}
      d_n & = \sum_i (-1)^i \cdot i \cdot \rk_\ZZ H^i_{W,c} (X, \ZZ (n)),\\
      \lambda (\zeta_X^* (n)^{-1})\cdot \ZZ & = \det\nolimits_\ZZ R\Gamma_{W,c} (X, \ZZ (n)).
    \end{align*}

  \item<4-> Note: this would imply
    $d_n = \sum_i (-1)^i \dim_\RR H_c^i (G_\RR, X (\CC), \RR (n))$.
  \end{itemize}
\end{frame}

% % % % % % % % % % % % % % % % % % % % % % % % % % % % % % % % % % % % % % % % % % % % %

\begin{frame}
  \frametitle{What it's good for?}

  \begin{itemize}
  \item<2-> If $X$ is proper and regular, then $\mathbf{C} (X,n)$ is equivalent
    to the conjecture of Flach and Morin.

  \item<3-> (Whenever makes sense) compatible with the
    \textbf{Tamagawa number conjecture} (Bloch--Kato--Fontaine--Perrin-Riou).

  \item<4-> Well-behaved under decompositions: for $Z \to X \leftarrow U$ holds
    $\zeta_X (s) = \zeta_Z (s) \cdot \zeta_U (s)$ (obviously), and in fact
    $$\mathbf{C} (X,n) \iff \mathbf{C} (Z,n) + \mathbf{C} (U,n).$$
  \end{itemize}
\end{frame}

% % % % % % % % % % % % % % % % % % % % % % % % % % % % % % % % % % % % % % % % % % % % %

\begin{frame}
  \frametitle{* Construction (after Flach and Morin)}

  \onslide<2->{Consider the étale sheaf
    $\ZZ (n) \dfn \bigoplus_p \varinjlim_r  j_{p!} \mu_{p^r}^{\otimes n} [-1]$,
    where $j_p\colon X [1/p] \hookrightarrow X$.}

  \vspace{1em}

  \onslide<3->{$\begin{tikzcd}[column sep=0.5em, font=\footnotesize, ampersand replacement=\&]
      \& \& R\Gamma_\text{\it W,c} (X, \ZZ (n))\ar{d} \\
      {R\!\Hom (R\Gamma (X_\text{\it ét}, \ZZ^c (n)), \QQ [-2])} \ar{r} \ar{d} \& R\Gamma_c (X_\text{\it ét}, \ZZ (n)) \ar{r}\ar{d}{\text{comparison}} \& R\Gamma_\text{\it fg} (X, \ZZ (n)) \ar{r}\ar[dashed]{d} \& {[+1]} \ar{d} \\
      0\ar{r} \& R\Gamma_c (G_\RR, X (\CC), \ZZ (n)) \ar{r}{\text{id}} \& R\Gamma_c (G_\RR, X (\CC), \ZZ (n))\ar{d}\ar{r} \& 0 \\
      \& \& R\Gamma_\text{\it W,c} (X, \ZZ (n)) [1]
    \end{tikzcd}$}
\end{frame}

% % % % % % % % % % % % % % % % % % % % % % % % % % % % % % % % % % % % % % % % % % % % %

\begin{frame}
  \frametitle{Some questions}

  \begin{itemize}
  \item<2-> A regulator for non-smooth $X_\CC$?

  \item<3-> A less ad-hoc definition of Weil-étale complexes?

    Morally, there should be a Grothendieck topology behind everything.
  \end{itemize}
\end{frame}

% % % % % % % % % % % % % % % % % % % % % % % % % % % % % % % % % % % % % % % % % % % % %

\begin{frame}[plain]
  \headingfont

  \begin{center}
    {\huge Thank you!}
  \end{center}
\end{frame}

\end{document}
