\ifdefined\handout
  \documentclass[handout]{beamer}
\else
  \documentclass{beamer}
\fi

\usetheme{boxes}
\definecolor{beamer@structure@color}{rgb}{0,0,0}

\usecolortheme{structure}

\setbeamertemplate{footline}[frame number]
\setbeamertemplate{frametitle}{\color{black}
\def\myhrulefill{\leavevmode\leaders\hrule height 1pt\hfill\kern 0pt}
\headingfont\insertframetitle\par\vskip-8pt\myhrulefill}

\usepackage{tikz-cd}
\usetikzlibrary{arrows}
\usetikzlibrary{calc}
\usetikzlibrary{babel}
\usetikzlibrary{decorations.pathmorphing,shapes}

\newcommand*{\longhookrightarrow}{\ensuremath{\lhook\joinrel\relbar\joinrel\rightarrow}}

\newcommand{\CC}{\mathbb{C}}
\newcommand{\FF}{\mathbb{F}}
\newcommand{\NN}{\mathbb{N}}
\newcommand{\PP}{\mathbb{P}}
\newcommand{\QQ}{\mathbb{Q}}
\newcommand{\RR}{\mathbb{R}}
\newcommand{\ZZ}{\mathbb{Z}}

\DeclareMathOperator{\fchar}{char}

\renewcommand{\AA}{\mathbb{A}}

\DeclareMathOperator{\Gal}{Gal}
\renewcommand{\Re}{\operatorname{Re}}
\newcommand{\dfn}{\mathrel{\mathop:}=}
\DeclareMathOperator{\coker}{coker}
\DeclareMathOperator{\Hom}{Hom}
\DeclareMathOperator{\ord}{ord}
\DeclareMathOperator{\Pic}{Pic}
\DeclareMathOperator{\rk}{rk}
\DeclareMathOperator{\Spec}{Spec}
\DeclareMathOperator{\vol}{vol}

\newcommand{\et}{\text{\it ét}}
\newcommand{\tors}{\text{\it tors}}
\newcommand{\Wc}{\text{\it W,c}}

\newcommand{\RHom}{R\!\Hom}

\setbeamertemplate{navigation symbols}{}

\usepackage{array}
\newcolumntype{x}[1]{>{\centering\hspace{0pt}}p{#1}}
\definecolor{shadecolor}{rgb}{0.89,0.89,0.89}
\usepackage{colortbl}

\newcommand{\term}{\textbf}

\usepackage{mathspec}
\setsansfont[BoldFont={IBM Plex Sans Bold}, ItalicFont={IBM Plex Sans Italic}]{IBM Plex Sans}
\setmathrm[BoldFont={IBM Plex Sans Bold}, ItalicFont={IBM Plex Sans Italic}]{IBM Plex Sans}
\newfontfamily\headingfont[]{IBM Plex Sans Bold}

\setbeamercovered{transparent=15}

\begin{document}

%%%%%%%%%%%%%%%%%%%%%%%%%%%%%%%%%%%%%%%%%%%%%%%%%%%%%%%%%%%%%%%%%%%%%%%%%%%%%%%%

\begin{frame}[noframenumbering]
  \begin{center}
    {\LARGE\bf Cohomología Weil-étale para n < 0

    }

    \vspace{3em}

    {\large\bf Alexey Beshenov}

    \vspace{1em}

    {\small (Centro de Investigación en Matemáticas, México)}

    \vspace{3em}

    09/02/2021

    \vspace{1em}

    Seminario de la teoría de números UAM-ICMAT

  \end{center}
\end{frame}

%%%%%%%%%%%%%%%%%%%%%%%%%%%%%%%%%%%%%%%%%%%%%%%%%%%%%%%%%%%%%%%%%%%%%%%%%%%%%%%%

\begin{frame}
  \frametitle{Plan de charla}

  \begin{enumerate}
  \item<2-> \textbf{Motivación}:
    funciones zeta aritméticas,
    valores especiales
    y su interpretación cohomológica.

  \item<3-> \textbf{Programa Weil-étale de Lichtenbaum}:
    ideas y resultados principales.

  \item<4-> \textbf{Mi trabajo}:
    conjeturas y resultados incondicionales.

  \item<5-> \textbf{Preguntas para el futuro}.
  \end{enumerate}
\end{frame}

%%%%%%%%%%%%%%%%%%%%%%%%%%%%%%%%%%%%%%%%%%%%%%%%%%%%%%%%%%%%%%%%%%%%%%%%%%%%%%%%

\begin{frame}[plain]
  \headingfont

  \begin{center}
    {\huge Motivación (motívica)}
  \end{center}
\end{frame}

%%%%%%%%%%%%%%%%%%%%%%%%%%%%%%%%%%%%%%%%%%%%%%%%%%%%%%%%%%%%%%%%%%%%%%%%%%%%%%%%

\begin{frame}
  \frametitle{Funciones zeta aritméticas\\y sus valores especiales}

  \begin{itemize}
  \item<2-> \textbf{Esquema aritmético} $X$ = separado, de tipo finito sobre
    $\Spec \ZZ$.

  \item<3-> \textbf{Función zeta}:
    \[ \begin{tikzcd}[ampersand replacement=\&, row sep=0.4em, decoration=snake]
        X\ar[->,decorate]{r} \& \zeta (X,s) = \prod\limits_{\substack{x \in X \\ \text{cerrado}}} \frac{1}{1 - \#\kappa (x)^{-s}}
      \end{tikzcd} \]

  \item<4-> Convergencia para $s > \dim X$.

  \item<5-> Conjetura: prolongación meromorfa a $s \in \CC$,

    ecuación funcional
    $\zeta (X,s) \leftrightarrow \zeta (X, \dim X - s)$.

  \item<6-> Fijemos $n \in \ZZ$.

  \item<7-> $\ord_{s=n} \zeta (X,s) = d_n \dfn \text{\textbf{orden de anulación} en }s = n$.

  \item<8-> \textbf{Valor especial}: $\zeta^* (X,n) \dfn \lim_{s \to n} (s-n)^{-d_n}\,\zeta (X,s)$.
  \end{itemize}
\end{frame}

%%%%%%%%%%%%%%%%%%%%%%%%%%%%%%%%%%%%%%%%%%%%%%%%%%%%%%%%%%%%%%%%%%%%%%%%%%%%%%%%

\begin{frame}
  \frametitle{Ejemplos extensivamente estudiados}

  \begin{itemize}
  \item<2-> \textbf{Función zeta de Dedekind} (siglo XIX).

    $F/\QQ$ cuerpo de números, $\mathcal{O}_F \subset F$ anillo de enteros.
    \[ \zeta_F (s) \dfn \zeta (\Spec \mathcal{O}_F, s) \stackrel{\text{Euler}}{=} \sum_{0 \ne \mathfrak{a} \subseteq \mathcal{O}_F} \frac{1}{\# (\mathcal{O}_F/\mathfrak{a})^s}. \]
    E.g. $\zeta_\QQ (s) = \zeta (\Spec \ZZ, s) = \zeta (s)$.

  \item<3-> \textbf{Función zeta de Hasse--Weil} (siglo XX).

    $X/\FF_q$ variedad sobre cuerpo finito.
    \[ Z (X,t) \dfn \exp \left(\sum_{k\ge 1} \frac{\# X (\FF_{q^k})}{k}\,t^k\right) \stackrel{\text{Dwork}}{\in} \QQ (t). \]

    \[ \zeta (X,s) = Z (X,q^{-s}). \]

    Conjeturas de Weil (Grothendieck, Deligne, \dots)
  \end{itemize}
\end{frame}

%%%%%%%%%%%%%%%%%%%%%%%%%%%%%%%%%%%%%%%%%%%%%%%%%%%%%%%%%%%%%%%%%%%%%%%%%%%%%%%%

\begin{frame}
  \frametitle{Fórmula del número de clases (Dirichlet)}

  \begin{itemize}
  \item<2-> $s = 0$.

  \item<3-> $\ord_{s = 0} \zeta_F (s) = r_1 + r_2 - 1 = \rk \mathcal{O}_F^\times$.

  \item<4-> $\zeta_F^* (0) = -\frac{\# \Pic (\mathcal{O}_F)}{\# (\mathcal{O}_F)^\times_\tors}\,R_F$.

  \item<5-> Similar para curvas proyectivas lisas $X/\FF_q$:

    $\ord_{s = 0} \zeta (X,s) = -1$ y
    $\zeta^* (X,0) = \frac{\# \Pic^0 (X)}{q-1}$.

  % \item $X = C/\FF_q$ curva proyectiva lisa de género $g$.

  %   \[ \zeta (X,s) = \frac{P (q^{-s})}{(1 - q^{-s})\,(1 - q^{1-s})}, \quad
  %     P (t) \in \ZZ [t], ~
  %     \deg P (t) = 2g. \]

  %   \[ \ord_{s = 0} \zeta (X,s) = -1 \]

  \item<6-> ¿Generalizaciones?
  \end{itemize}
\end{frame}

%%%%%%%%%%%%%%%%%%%%%%%%%%%%%%%%%%%%%%%%%%%%%%%%%%%%%%%%%%%%%%%%%%%%%%%%%%%%%%%%

\begin{frame}
  \frametitle{Cohomología motívica étale}

  \begin{itemize}
  \item<2-> Lichtenbaum, 1984: complejos hipotéticos (!) de haces sobre $X_\et$
    responsables por los valores especiales.

  \item<3-> Bloch, 1986: complejos de ciclos / grupos de Chow superiores.

  \item<4-> Versión étale: complejo de haces $\ZZ^c (n)$ sobre $X_\et$.

  \item<5-> Funciona para $X / \Spec \ZZ$ (Levine, Geisser, \ldots).

  \item<6-> Para $X$ propio, regular, $d = \dim X$:
    \[ \underbrace{H^i (X_\et, \ZZ^c (n))}_{\text{coh. de Borel--Moore motívica}} = \underbrace{H^{i+2d} (X_\et, \ZZ (d-n))}_{\text{coh. motívica habitual}}. \]

  \item<7-> Pocos cálculos explícitos disponibles.

  \item<8-> Generación finita --- ???
  \end{itemize}
\end{frame}

%%%%%%%%%%%%%%%%%%%%%%%%%%%%%%%%%%%%%%%%%%%%%%%%%%%%%%%%%%%%%%%%%%%%%%%%%%%%%%%%

\begin{frame}
  \frametitle{Conjetura cohomológica de Lichtenbaum}

  \begin{itemize}
  \item<2-> $n \le 0$.

  \item<3-> $d_n = \ord_{s = n} \zeta_F (s) = \begin{cases}
      r_1 + r_2, & n < 0\text{ par}, \\
      r_1, & n < 0\text{ impar}.
    \end{cases}$

  \item<4-> $\ZZ^c (0) = \mathbb{G}_m [1]$, $\zeta_F^* (0) = -\frac{\# H^1 (X_\et, \mathbb{G}_m)}{\# H^0 (X_\et, \mathbb{G}_m)_\tors}\,R_F = -\frac{\# H^0 (X_\et, \ZZ^c (0))}{\# H^{-1} (X_\et, \ZZ^c (0))_\tors}\,R_F.$

  \item<5-> \textbf{Conjetura}: para $n \le 0$
    \[ \zeta_F^* (n) = \pm\frac{\# H^0 (X_\et, \ZZ^c (n))}{\# H^{-1} (X_\et, \ZZ^c (n))_\tors}\,R_{F,n}. \]

  \item<6-> En términos de $K_i (\mathcal{O}_F)$, para $F$ real, $n$ impar
    ($R_{F,n} = 1$):

    Lichtenbaum, 1973.

  \item<7-> \textbf{Reguladores superiores}: Borel, Beilinson:
    \[ R_{F,n} = \vol\coker \Bigl(\underbrace{H^{-1} (X_\et, \ZZ^c(n))}_{\rk_\ZZ = d_n} \to \underbrace{H^1_\mathcal{D} (G_\RR, X(\CC), \RR(n))}_{\dim_\RR = d_n}\Bigr). \]

  \item<8-> Teorema para $F/\QQ$ abeliano (¡mediante TNC!).
  \end{itemize}
\end{frame}

%%%%%%%%%%%%%%%%%%%%%%%%%%%%%%%%%%%%%%%%%%%%%%%%%%%%%%%%%%%%%%%%%%%%%%%%%%%%%%%%

\iffalse
\begin{frame}
  \frametitle{Caso de curvas}

  \begin{itemize}
  \item $X = C/\FF_q$ cualquier curva.

  \item $\ord_{s = n} \zeta (X,s) = 0$ para $n < 0$.

  \item $\zeta (X,n) = \pm \frac{|H^0 (X_\et, \ZZ^c (n))|}{|H^{-1} (X_\et, \ZZ^c (n))| \cdot |H^1 (X_\et, \ZZ^c (n))|}$.

  \item Ejemplo singular: cúbica nodal $X = \PP^1_{\FF_q} / (0\sim 1)$.

    \begin{align*}
      H^{-1} (X_\et, \ZZ^c (n)) & = \ZZ/(q^{1-n} - 1), \\
      H^{0,1} (X_\et, \ZZ^c (n)) & = \ZZ/(q^{-n} - 1).
    \end{align*}

    \[ \zeta (X,s) = \frac{1}{1 - q^{1-s}}. \]
  \end{itemize}
\end{frame}
\fi

%%%%%%%%%%%%%%%%%%%%%%%%%%%%%%%%%%%%%%%%%%%%%%%%%%%%%%%%%%%%%%%%%%%%%%%%%%%%%%%%

\begin{frame}[plain]
  \headingfont

  \begin{center}
    {\huge Cohomología Weil-étale}
  \end{center}
\end{frame}

%%%%%%%%%%%%%%%%%%%%%%%%%%%%%%%%%%%%%%%%%%%%%%%%%%%%%%%%%%%%%%%%%%%%%%%%%%%%%%%%

\begin{frame}
  \frametitle{Estructura de la cohomología motívica\\para $X/\ZZ$ (Lichtenbaum)}

  \begin{itemize}
  \item<2-> Conjeturalmente (!)
    \[ H^i (X_\et, \ZZ^c (n)) = \begin{cases}
        \text{finitamente generado}, & i \le -2n, \\
        \text{finito}, & i = -2n + 1, \\
        \text{tipo cofinito}, & i \ge -2n + 2.
    \end{cases} \]

  \item<3-> Tipo cofinito = $\QQ/\ZZ$-dual a finitamente generado.

    Manifestación de dualidad aritmética (Artin--Verdier, \ldots).

  \item<4-> * si $n < 0$, entonces $H^i (X_\et, \ZZ^c (n))$ son finitamente generados.

  \item<5-> Conjetura de Beilinson--Soulé: $H^i (X_\et, \ZZ^c (n)) = 0$ para
    $i < -2\dim X$.

  \item<6-> En general, $H^i (X_\et, \ZZ^c (n)) \ne 0$ para $i \gg 0$.
  \end{itemize}
\end{frame}

%%%%%%%%%%%%%%%%%%%%%%%%%%%%%%%%%%%%%%%%%%%%%%%%%%%%%%%%%%%%%%%%%%%%%%%%%%%%%%%%

\begin{frame}
  \frametitle{Estructura de la cohomología motívica\\para $X/\FF_q$ (Lichtenbaum)}

  \begin{itemize}
  \item<2-> Conjeturalmente (!)
    \[ H^i (X_\et, \ZZ^c (n)) = \begin{cases}
        \text{finito}, & i \ne -2n, \, -2n+2, \\
        \text{finitamente generado}, & i = -2n, \\
        \text{tipo cofinito}, & i = -2n + 2.
      \end{cases} \]

  \item<3-> * si $n < 0$, entonces $H^i (X_\et, \ZZ^c (n))$ son finitos.
  \end{itemize}
\end{frame}

%%%%%%%%%%%%%%%%%%%%%%%%%%%%%%%%%%%%%%%%%%%%%%%%%%%%%%%%%%%%%%%%%%%%%%%%%%%%%%%%

\begin{frame}
  \frametitle{Cohomología Weil-étale (Lichtenbaum)}

  \begin{itemize}
  \item<2-> $\text{Cohomología motívica étale} \rightsquigarrow
    \text{cohomología Weil-étale}$.

  \item<3-> Grupos $H^i_\Wc (X,\ZZ(n))$ finitamente generados, nulos para
    $i \gg 0$.

  \item<4-> Sucesión exacta
    $$\cdots \to H^i_\Wc (X,\ZZ(n)) \otimes \RR \xrightarrow{\smile\theta} H^{i+1}_\Wc (X,\ZZ(n)) \otimes \RR \to \cdots$$

  \item<5-> $H^i_\Wc (X, \ZZ(n))$ codifica $\ord_{s=n} \zeta (X,s)$ y $\zeta^* (X,n)$.

    (¡Detalles más adelante!)

  % \item ¿Por qué Weil-étale?
  %   Construcción para $X/\FF_q$:
  %   \[ R\Gamma (G, R\Gamma (X_{\overline{\FF}_q,\et}, \ZZ^c (n))). \]

  %   $G \subset \Gal (\overline{\FF}_q/\FF_q)$ = grupo de Weil, generado por el
  %   Frobenius ($\cong \ZZ \subset \widehat{\ZZ}$).

  % \item Topos Weil-étale (?)
  \end{itemize}
\end{frame}

%%%%%%%%%%%%%%%%%%%%%%%%%%%%%%%%%%%%%%%%%%%%%%%%%%%%%%%%%%%%%%%%%%%%%%%%%%%%%%%%

\begin{frame}
  \frametitle{Algunos resultados}

  \begin{itemize}
  \item<2-> «Resultado» =
    \begin{itemize}
    \item<3-> definir $H^i_\Wc (X, \ZZ(n))$ asumiendo las conjeturas de
      Lichtenbaum sobre estructura de cohomología motívica,
    \item<4-> formular la relación conjetural de $H^i_\Wc (X, \ZZ(n))$ con
      $\ord_{s=n} \zeta (X,s)$ y $\zeta^* (X,n)$,
    \item<5-> establecer relaciones con otras conjeturas, probar casos
      particulares.
    \end{itemize}

  \item<6-> Lichtenbaum (2005): $X/\FF_q$.

  \item<7-> Geisser (2004--\dots): $X/\FF_q$, posiblemente singular.

  \item<8-> Lichtenbaum (2009): $X = \Spec \mathcal{O}_F$.

  \item<9-> Morin (2014): $X/\ZZ$ propio y regular, $n = 0$.

  \item<10-> Flach, Morin (2018): \rule[0.25em]{2.3cm}{0.6pt}, $n \in \ZZ$.

  \item<11-> B. (2020/21): cualquier esquema aritmético $X/\ZZ$, $n < 0$.
  \end{itemize}
\end{frame}

%%%%%%%%%%%%%%%%%%%%%%%%%%%%%%%%%%%%%%%%%%%%%%%%%%%%%%%%%%%%%%%%%%%%%%%%%%%%%%%%

\begin{frame}[plain]
  \headingfont

  \begin{center}
    {\huge Mi trabajo}
  \end{center}
\end{frame}

%%%%%%%%%%%%%%%%%%%%%%%%%%%%%%%%%%%%%%%%%%%%%%%%%%%%%%%%%%%%%%%%%%%%%%%%%%%%%%%%

\begin{frame}
  \frametitle{Complejos Weil-étale}

  \begin{itemize}
  \item<2-> $X \to \Spec \ZZ$ separado, de tipo finito, $n < 0$.

  \item<3-> Asumamos $\mathbf{L}^c (X_\et, n)$: los grupos
    $H^i (X_\et, \ZZ^c(n))$ son finitamente generados para todo $i \in \ZZ$.

  \item<4-> Existe complejo $R\Gamma_\Wc (X, \ZZ(n))$.

  \item<5-> $H^i_\Wc (X, \ZZ(n))$ son finitamente generados, nulos para
    $i \notin [0, 2\dim X + 1]$.

  % \item Si $X/\FF_q$, entonces hay isomorfismo de grupos finitos
  %   \begin{align*}
  %     H^i_\Wc (X, \ZZ (n)) & \cong \Hom (H^{2-i} (X_\et, \ZZ^c (n)), \QQ/\ZZ) \\
  %                          & \cong H^{i-1}_c (X_\et, \QQ/\ZZ' (n)), \quad \QQ/\ZZ' (n) = \varinjlim_{p\nmid m} \mu_m^{\otimes n}.
  %   \end{align*}

  \item<6-> Se escinde con coeficientes racionales/reales:
    \[ R\Gamma_\Wc (X, \ZZ(n)) \otimes \RR \cong
      \begin{array}{c}
        \RHom(R\Gamma(X_\et, \ZZ^c(n)), \RR)[-1]\\
        \oplus \\
        R\Gamma_c (G_\RR, X(\CC), \RR(n))[-1]
      \end{array} \]

  \item<7-> $\RR(n) \dfn (2\pi i)^n\,\RR$, $G_\RR \dfn \Gal (\CC/\RR)$.
  \end{itemize}
\end{frame}

%%%%%%%%%%%%%%%%%%%%%%%%%%%%%%%%%%%%%%%%%%%%%%%%%%%%%%%%%%%%%%%%%%%%%%%%%%%%%%%%

\begin{frame}
  \frametitle{Ingrediente principal de la construcción}

  \begin{itemize}
  \item<2-> Dualidad aritmética
    \[ \Hom (\underbrace{H^{2-i} (X_\et, \ZZ^c (n))}_{\text{finitamente generado}}, \QQ/\ZZ) \cong
      \underbrace{\widehat{H}^i_c (X_\et, \ZZ' (n))}_{\text{tipo cofinito}}, \]

  \item<3-> $\ZZ' (n) = \QQ/\ZZ' (n) [-1] = \bigoplus_p \varinjlim_r j_{p!} \mu_{p^r}^{\otimes n} [-1]$,
    \quad $j_p\colon X[1/p] \hookrightarrow X$.

  \item<4-> $\widehat{H}^i_c$ = cohomología modificada, toma en cuenta $X (\RR)$.

  \item<5-> Generalización de la dualidad de Artin--Verdier para
    $X = \Spec \mathcal{O}_F$.
  \end{itemize}
\end{frame}

%%%%%%%%%%%%%%%%%%%%%%%%%%%%%%%%%%%%%%%%%%%%%%%%%%%%%%%%%%%%%%%%%%%%%%%%%%%%%%%%

\begin{frame}
  \frametitle{Regulador}

  \begin{itemize}
  \item<2-> Asumamos que la fibra $X_\CC$ es lisa.

  \item<3-> Construcción de Kerr--Lewis--Müller-Stach $\Longrightarrow$
    \[ Reg\colon R\Gamma (X_\et, \RR^c (n)) \to
      \RHom (R\Gamma_c (G_\RR, X(\CC), \RR (n)), \RR[1]). \]

  \item<4-> * La llegada no es la (co)homología de Deligne--Beilinson, sino
    simplemente $H_c^i (G_\RR, X(\CC), \RR (n))^\vee$, porque $n < 0$.

  \item<5-> Conjetura $\mathbf{B} (X,n)$ (Beilinson):
    \[ Reg^\vee\colon R\Gamma_c (G_\RR, X (\CC), \RR (n)) [-1] \to
      \RHom (R\Gamma (X_\et, \ZZ^c (n)), \RR) \]
    es un cuasi-isomorfismo.
  \end{itemize}
\end{frame}

%%%%%%%%%%%%%%%%%%%%%%%%%%%%%%%%%%%%%%%%%%%%%%%%%%%%%%%%%%%%%%%%%%%%%%%%%%%%%%%%

\begin{frame}
  \frametitle{Conjetura del orden de anulación}

  \begin{itemize}
  \item<2-> $\mathbf{VO} (X,n)$: asumiendo $\mathbf{L}^c (X,n)$,
    \[ \tag{*} \ord_{s=n} \zeta (X,s) =
      \sum_{i \in \ZZ} (-1)^i \cdot i \cdot \rk_\ZZ H^i_\Wc (X, \ZZ (n)). \]

  \item<3-> Asumiendo $\mathbf{B} (X,n)$,
    \begin{align}
      \tag{**} \ord_{s=n} \zeta (X,s) & = \sum_{i \in \ZZ} (-1)^i \dim_\RR H^i_c (X(\CC), \RR (n))^{G_\RR} \\
      \tag{***} & = \sum_{i \in \ZZ} (-1)^{i+1} \rk_\ZZ H^i (X_\et, \ZZ^c (n)).
    \end{align}

  \item<4-> (**) concuerda con la ecuación funcional (conjetural).

    Para $n < 0$ polos y ceros vienen de los $\Gamma$-factores.

  \item<5-> (***) concuerda con una conjetura de Soulé (1984).
  \end{itemize}
\end{frame}

%%%%%%%%%%%%%%%%%%%%%%%%%%%%%%%%%%%%%%%%%%%%%%%%%%%%%%%%%%%%%%%%%%%%%%%%%%%%%%%%

\begin{frame}
  \frametitle{Ejemplo de juguete}

  \begin{itemize}
  \item<2-> $X = \Spec \mathcal{O}_F$. Espacio $X (\CC)$ con $G_\RR$-acción:
    \[ \begin{tikzpicture}[ampersand replacement=\&]
        \matrix(m)[matrix of math nodes, row sep=0.75em, column sep=0.75em,
        text height=1ex, text depth=0.2ex]{
          ~ \& ~ \& ~ \& ~ \& ~ \& \bullet \& \bullet \& \cdots \& \bullet \\
          \bullet \& \bullet \& \cdots \& \bullet \\
          ~ \& ~ \& ~ \& ~ \& ~ \& \bullet \& \bullet \& \cdots \& \bullet \\};

        \draw[->] (m-2-1) edge[loop above,min distance=10mm] (m-2-1);
        \draw[->] (m-2-2) edge[loop above,min distance=10mm] (m-2-2);
        \draw[->] (m-2-4) edge[loop above,min distance=10mm] (m-2-4);

        \draw[->] (m-1-6) edge[bend left] (m-3-6);
        \draw[->] (m-1-7) edge[bend left] (m-3-7);
        \draw[->] (m-1-9) edge[bend left] (m-3-9);

        \draw[->] (m-3-6) edge[bend left] (m-1-6);
        \draw[->] (m-3-7) edge[bend left] (m-1-7);
        \draw[->] (m-3-9) edge[bend left] (m-1-9);

        \draw ($(m-3-1)!.5!(m-3-4)$) node[yshift=-2em,anchor=base] {$r_1$ puntos};
        \draw ($(m-3-6)!.5!(m-3-9)$) node[yshift=-2em,anchor=base] {$2\,r_2$ puntos};
      \end{tikzpicture} \]

  \item<3-> Complejo $R\Gamma_c (X (\CC), \RR (n))$:
    $$\RR (n)^{\oplus r_1} \oplus (\RR (n) \oplus \RR (n))^{r_2},$$
    $G_\RR$-acción por $z \mapsto \overline{z}$ vs. $(z,w) \mapsto (\overline{w}, \overline{z})$.

  \item<4-> $\ord_{s=n} \zeta_F (s) =
    \dim_\RR H^0_c (X (\CC), \RR (n))^{G_\RR} =
    \rk_\ZZ H^{-1}_\et (X, \ZZ^c (n)) =
    \left\{\begin{array}{ll}
      r_1 + r_2, & n \text{ par},\\
      r_2, & n \text{ impar}.
    \end{array}\right\}$
  \end{itemize}
\end{frame}

%%%%%%%%%%%%%%%%%%%%%%%%%%%%%%%%%%%%%%%%%%%%%%%%%%%%%%%%%%%%%%%%%%%%%%%%%%%%%%%%

\begin{frame}
  \frametitle{Determinantes de complejos}

  \begin{itemize}
  \item<2-> Para módulos proyectivos finitamente generados:

    $\det\nolimits_R P \dfn \bigwedge^{\rk P} P$

    (invertible = proyectivo de rk 1).

  \item<3-> Funtor
    \[ \left(\!\!\begin{array}{c}
              \text{módulos proyectivos f.g.}, \\
              \text{isomorfismos}
       \end{array}\!\!\right) \rightsquigarrow
       \left(\!\!\begin{array}{c}
               \text{módulos invertibles}, \\
               \text{isomorfismos}
       \end{array}\!\!\right). \]

  \item<4-> Knudsen, Mumford, 1976: extensión
    \[ \left(\!\!\begin{array}{c}
              \text{complejos perfectos}, \\
              \text{cuasi-isomorfismos}
       \end{array}\!\!\right) \rightsquigarrow
       \left(\!\!\begin{array}{c}
                \text{módulos invertibles}, \\
                \text{isomorfismos}
       \end{array}\!\!\right). \]

 \item<5-> $\det\nolimits_R A^\bullet \cong \bigotimes_{i\in \ZZ} (\det_R H^i (A^\bullet))^{(-1)^i}$,
   $\det\nolimits_R 0 \cong R$.

  \item<6-> Compatibilidad con cambio de base.
  \end{itemize}
\end{frame}

%%%%%%%%%%%%%%%%%%%%%%%%%%%%%%%%%%%%%%%%%%%%%%%%%%%%%%%%%%%%%%%%%%%%%%%%%%%%%%%%

\begin{frame}
  \frametitle{Morfismo de trivialización}

  \begin{itemize}
  \item<2-> Cuasi-isomorfismo de complejos, asumiendo $\mathbf{B} (X,n)$:
    \[ \begin{tikzcd}[column sep=4em,ampersand replacement=\&]
        \begin{array}{c} R\Gamma_c (G_\RR, X (\CC), \RR (n)) [-2] \\ \oplus \\ R\Gamma_c (G_\RR, X (\CC), \RR (n)) [-1] \end{array} \ar{d}{Reg^\vee [-1] \oplus id}[swap]{\cong} \&[-2em] \text{\small * $\det\nolimits_R (A^\bullet \oplus A^\bullet [1]) \cong R$} \\
        \begin{array}{c} \RHom (R\Gamma (X_\et, \ZZ^c (n)), \RR) [-1] \\ \oplus \\ R\Gamma_c (G_\RR, X (\CC), \RR (n)) [-1] \end{array} \ar{r}{\text{escisión}}[swap]{\cong} \& R\Gamma_\Wc (X, \ZZ (n)) \otimes \RR
      \end{tikzcd} \]

  \item<3-> Isomorfismo de determinantes:
    \[ \lambda\colon \RR \xrightarrow{\cong}
      \det\nolimits_\RR \Bigl(R\Gamma_\Wc (X, \ZZ (n)) \otimes \RR\Bigr) \cong
      \Bigl(\det\nolimits_\ZZ R\Gamma_\Wc (X, \ZZ (n))\Bigr) \otimes \RR. \]
  \end{itemize}
\end{frame}

%%%%%%%%%%%%%%%%%%%%%%%%%%%%%%%%%%%%%%%%%%%%%%%%%%%%%%%%%%%%%%%%%%%%%%%%%%%%%%%%

\begin{frame}
  \frametitle{Conjetura del valor especial}

  \begin{itemize}
  \item<2-> Definimos
  \[ \lambda\colon \RR \xrightarrow{\cong}
    (\underbrace{\det\nolimits_\ZZ R\Gamma_\Wc (X, \ZZ (n))}_{\ZZ\text{-módulo de rk }1}) \otimes \RR. \]

  \item<3-> Asumamos
    \begin{itemize}
    \item $\mathbf{L}^c (X_\et, n)$: generación finita de $H^i (X_\et, \ZZ^c(n))$,
    \item fibra $X_\CC$ lisa,
    \item $\mathbf{B} (X,n)$: regulador,
    \item prolongación meromorfa alrededor de $s = n < 0$.
    \end{itemize}

  \item<4-> $\mathbf{C} (X,n)$: el valor especial es $s = n$ se determina salvo
    signo por
    \[ \lambda (\zeta^* (X,n)^{-1})\cdot \ZZ =
      \det\nolimits_\ZZ R\Gamma_\Wc (X, \ZZ (n)). \]
  \end{itemize}
\end{frame}

%%%%%%%%%%%%%%%%%%%%%%%%%%%%%%%%%%%%%%%%%%%%%%%%%%%%%%%%%%%%%%%%%%%%%%%%%%%%%%%%

\begin{frame}
  \frametitle{Caso de variedades sobre cuerpos finitos}

  \begin{itemize}
  \item<2-> $\mathbf{C} (X,n)$ es equivalente a la fórmula
    \[ \zeta (X,n) = \prod_{i \in \ZZ} |H^i (X_\et, \ZZ^c (n))|^{(-1)^i}. \]

  \item<3-> Ejemplo singular: cúbica nodal $X = \PP^1_{\FF_q} / (0\sim 1)$.
    \begin{align*}
      H^{-1} (X_\et, \ZZ^c (n)) & = \ZZ/(q^{1-n} - 1), \\
      H^{0,1} (X_\et, \ZZ^c (n)) & = \ZZ/(q^{-n} - 1).
    \end{align*}

    \[ \zeta (X,s) = \frac{1}{1 - q^{1-s}}. \]

  \item<4-> $\mathbf{C} (X,n)$ se cumple incondicionalmente, asumiendo
    $\mathbf{L}^c (X_\et, n)$, si $X/\FF_q$ es lisa y proyectiva.

  \item<5-> Se cumple para cualquier $X$, asumiendo resolución de
    singularidades sobre $\FF_q$ (!!)
  \end{itemize}
\end{frame}

%%%%%%%%%%%%%%%%%%%%%%%%%%%%%%%%%%%%%%%%%%%%%%%%%%%%%%%%%%%%%%%%%%%%%%%%%%%%%%%%

\begin{frame}
  \frametitle{Compatibilidades}

  \begin{itemize}
  \item<2-> \textbf{Uniones disjuntas}: si
    $X = \coprod_{1 \le i \le r} X_i$, entonces
    \[ \zeta (X,s) = \prod_{1 \le i \le r} \zeta (X_i,s). \]

  \item<3-> De acuerdo con esto,
    \begin{align*}
      \mathbf{VO} (X,n) & \iff \mathbf{VO} (X_i,n)\text{ para todo }i, \\
      \mathbf{C} (X,n) & \iff \mathbf{C} (X_i,n)\text{ para todo }i.
    \end{align*}

  \item<4-> \textbf{Descomposiciones cerrado-abiertas}:
    para $Z \not\hookrightarrow X \hookleftarrow U$,
    \[ \zeta (X,s) = \zeta (Z,s) \cdot \zeta (U,s). \]

  \item<5-> Dos de las tres conjeturas
    $\mathbf{VO} (X,n)$, $\mathbf{VO} (Z,n)$, $\mathbf{VO} (U,n)$
    (resp. $\mathbf{C} (X,n)$, $\mathbf{C} (Z,n)$, $\mathbf{C} (U,n)$)
    implican la tercera.

  \item<6-> \textbf{Fibrados afines}:
    $\zeta (\AA^r_X, s) = \zeta (X, s-r)$.

  \item<7-> $\mathbf{VO} (\AA^r_X, n) \iff \mathbf{VO} (X, n-r)$,
    $\mathbf{C} (\AA^r_X, n) \iff \mathbf{C} (X, n-r)$.
  \end{itemize}
\end{frame}

%%%%%%%%%%%%%%%%%%%%%%%%%%%%%%%%%%%%%%%%%%%%%%%%%%%%%%%%%%%%%%%%%%%%%%%%%%%%%%%%

\begin{frame}
  \frametitle{Aplicación: resultados nuevos incondicionales}

  \begin{itemize}
  \item<2-> Esquema \textbf{celular} $X \to B$: admite filtración por
    cerrados
    \[ X = Z_N \supseteq Z_{N-1} \supseteq \cdots \supseteq Z_0 \supseteq Z_{-1} = \emptyset, \]
    donde $Z_i\setminus Z_{i-1} \cong \coprod_j \AA^{r_{i_j}}_B$

  \item<3-> \textbf{Teorema} (B.): Sea $B$ un esquema aritmético
    $1$-dimensional. Asumamos que para todo punto genérico $\eta \in B$ se
    cumple uno de los dos:
    \begin{enumerate}
    \item[a)] $\fchar \kappa (\eta) = p > 0$;

    \item[b)] $\fchar \kappa (\eta) = 0$ y $\kappa (\eta)/\QQ$ es un cuerpo de
      números abeliano.
    \end{enumerate}

    Entonces, $\mathbf{VO} (X,n)$ y $\mathbf{C} (X,n)$ se cumplen para todo
    $n < 0$ y todo esquema aritmético $B$-celular $X$ con la fibra $X_\CC$ lisa.

  \item<4-> Idea: $\mathbf{C} (X,n)$ se conoce para curvas y cuerpos de números
    abelianos $F/\QQ$ (¡via TNC!). Proceder por inducción usando las
    compatibilidades.
  \end{itemize}
\end{frame}

%%%%%%%%%%%%%%%%%%%%%%%%%%%%%%%%%%%%%%%%%%%%%%%%%%%%%%%%%%%%%%%%%%%%%%%%%%%%%%%%

\begin{frame}
  \frametitle{Algunas preguntas para el futuro}

  \begin{itemize}
  \item<2-> El regulador de Kerr--Lewis--Müller-Stach está definido para la
    fibra $X_\CC$ lisa.

    ¿Cómo extenderlo al caso singular y conectar a esta maquinaria aritmética?

  \item<3-> Cuando la comparación tiene sentido, $\mathbf{C} (X,n)$ es
    equivalente a la TNC.

    ¿Cómo formular un análogo equivariante compatible con la ETNC?
  \end{itemize}
\end{frame}

%%%%%%%%%%%%%%%%%%%%%%%%%%%%%%%%%%%%%%%%%%%%%%%%%%%%%%%%%%%%%%%%%%%%%%%%%%%%%%%%

\begin{frame}[plain]
  \headingfont

  \begin{center}
    {\huge ¡Gracias por su atención!}
  \end{center}
\end{frame}

\end{document}
